\chapter{Introduction}

In recent years, the advent of deep learning has ushered in a transformative era in the expansive domain of computer vision, representing a paradigm shift in the way machines perceive and comprehend visual information. The marriage of artificial intelligence and neural network architectures, particularly the deep neural networks with their multi-layered structures, has fundamentally altered the landscape of visual processing. The essence of deep learning lies in its capacity to autonomously learn hierarchical representations from raw data, making it a potent tool for tackling complex visual tasks that were once deemed impossible to tackle. The ramifications of this paradigm extend beyond mere image recognition, permeating diverse sectors, including robotics, healthcare, security, and entertainment.

Within the purview of deep learning for computer vision, the journey embarked with a watershed moment in 2012 when \cite{krizhevsky2012imagenet} presented their work, currently known as AlexNet, a pioneering convolutional neural network (CNN) architecture. The proposed architecture heralded a new era by significantly outperforming traditional methods in the ImageNet Large Scale Visual Recognition Challenge~\citep{ILSVRC15}. This momentous achievement not only demonstrated the efficacy of deep learning in image classification but also sparked an exponential surge in research and development within the field. Subsequent architectures, including VGGNet, GoogLeNet, ResNet, and ViT became instrumental in pushing the boundaries of accuracy and scalability, solidifying deep learning's position as the cornerstone of modern computer vision.